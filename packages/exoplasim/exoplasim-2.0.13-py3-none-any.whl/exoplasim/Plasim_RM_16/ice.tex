\newtheorem{fig}{\vspace{-10pt} Figure}
\newtheorem{tab}{\hspace{-5pt} Table}

\def \dt{\Delta t}
\def \dz{\Delta z}
\def \eps{\varepsilon}
\def \degrees{^\circ}

\def \bc{\begin{center}}
\def \ec{\end{center}}
\def \be{\begin{equation}}
\def \ee{\end{equation}}
\def \ba{\begin{array}}
\def \ea{\end{array}}
\def \bt{\vspace{2mm} \begin{tabular}}
\def \et{\end{tabular}}
\def \bd{\begin{displaymath}}
\def \ed{\end{displaymath}}
\def \bi{\begin{itemize}}
\def \ei{\end{itemize}}
\def \ben{\begin{enumerate}}
\def \een{\end{enumerate}}
\def \bc{\begin{center}}
\def \ec{\end{center}}
\def \di{\displaystyle}

\def \bgf{\begin{figure} \bc}
\def \bgfh{\begin{figure}[h] \bc}
\def \bgft{\begin{figure}[t] \bc}
\def \bgfb{\begin{figure}[b] \bc}
\def \bgfht{\begin{figure}[ht] \bc}
\def \bgfhb{\begin{figure}[hb] \bc}
\def \ef{\ec \end{figure}}

\def \btb{\begin{table} \bc}
\def \btbh{\begin{table}[h] \bc}
\def \btbht{\begin{table}[ht] \bc}
\def \etb{\ec \end{table}}

\chapter{Model Description}
The sea ice model is based on the zero layer model of
 \cite{semtner1976}. This model
 computes the thickness of the sea ice from the thermodynamic
 balances at the top and the bottom of the sea ice.
 The zero layer assumes the temperature gradient in the ice to
 be linear and eliminates the capacity of the ice to store heat.
 Nevertheless, it has been used successfully in areas where ice
 is mostly seasonal and thus relatively thin ($\rm <\, 1\, m$)
 \cite{beckmann2001}.
 Thus, the model is expected to perform better in the Southern
 Ocean than in the Arctic, where multiyear, thick ice dominates. Sea ice is formed if the ocean
 temperature drops below the freezing point
 (set to 271.25 K) and is melted whenever the ocean temperature increases above this point. The prognostic
 variables are the sea ice temperature $T_i$, the ice
 thickness $h_i$ and the ice concentration $A$, which
 in the present model is boolean: A given grid point is either
 ice free ($A=0$) or ice covered ($A=1$). 

Freezing and melting of sea ice releases just the right amount of latent heat 
of fusion to close the energy balance with respect to the total heat flux
$Q$ in the mixed layer \cite{parkinson1979}:

\be
Q\, +\rho_i\, L_i\, \frac{\partial h_i}{\partial t} =\, 0,
\label{hieq}
\ee

where $\rho_i$ is the density of sea ice
 and $L_i$ denotes the latent heat
 of fusion of sea ice. Standard parameter values are given in
 Table \ref{iceparatab}. \cite{parkinson1979} Thus, the prognostic
 equation for the sea ice thickness is given as

\be
{\di \frac{\partial h_i}{\partial t} = \frac{-Q}{\rho_i \, L_i}.}
\label{hi_eq}
\ee

It is assumed that melting of sea ice takes place from above only, while 
freezing takes place at the lower side of the ice floe.


\section*{Basic equations}
In the presence of sea ice, the heat fluxes are defined as follows.
The total heat flux $Q\, \rm (W\, m^{-2})$ is given as
\be
Q \,=\, Q_a\, +Q_c\, +Q_o\, +\tilde{Q},
\ee
where $Q_a$ is the atmospheric heat flux,
$Q_c$ is the conductive heat flux through the ice, $Q_o$ denotes the oceanic 
heat
flux and $\tilde{Q}$ is the flux correction. The atmospheric heat flux

\be
Q_a = \left\{ \ba{lcr}
\, F_T\, +L\, +R_{s,\downarrow}\, +R_{s,\uparrow}\,
	+R_{l,\downarrow}\, +R_{l,\uparrow}\, & {\rm if} & T_s > T_f, \\
0 & {\rm if} &  T_s \le T_f.
\ea \right. \label{qa_eq}
\ee

is the sum of sensible ($F_T$) and latent heat flux ($L$), the incoming and
reflected short wave radiation ($R_{s,\downarrow}\, R_{s,\uparrow}$)
and the long wave radiation ($R_l$). It is set to zero in the case of freezing,
where the conductive heat flux applies (see below).
The conductive heat flux through the ice

\be
Q_c = \left\{ \ba{lcr}
0 & {\rm if} & T_s > T_f, \\
{\di \frac{\bar{\kappa}}{h_i\, +h_s}\, (T_s -T_f)} & {\rm if} &  T_s \le T_f.
\ea \right. \label{qc_eq}
\ee

is set to zero in the case of melting ice, as the ice melts at the top.
 $\bar{\kappa}$ is the mean conductivity
 of the sea ice floe and snow cover (with depth = $h_s$), computed as

\be 
{\di \bar{\kappa}\, =\, \frac{\kappa_i h_i\, +\kappa_s h_s}{h_i\, +h_s}}.
\label{eq_kap}
\ee

where $\kappa_i$ and $\kappa_s$ are the conductivities of sea ice and snow, respectively.

Commonly, the oceanic heat flux $Q_o$ is parameterized in terms 
of the difference between the freezing temperature and the temperature 
of the ocean (mixed layer or deep ocean). $Q_o$ sets an upper value for 
the ice thickness and, thus, limits the ice growth.  However, to avoid 
artificial sources or sinks of heat the oceanic heat flux $Q_o$ is set 
to zero in the present model. The ice thickness is limited to a prescribed 
value $h_{max}$ (default = 9m) by setting $\bar{\kappa} = 0$ 
(i.e. $Q_c$ = 0) for $h_i > h_{max}$. 

The flux correction $\tilde{Q}$, if applied, is used to  
constrain the sea ice to a given distribution. 
It is obtained from the (monthly) climatology of an uncoupled 
(prescribed ice) simulation as

\be
\tilde{Q}=<\rho_i L_i \frac{h_{clim}-h_i}{\Delta t}>
\ee

where  $h_{clim}$ is the (prescribed) climatological ice, $\Delta t$ 
is the models time step and $<...>$ denotes a climatological (monthly) average.

In the case of melting, the ice thickness may become negative if the 
energy available for melting is greater than needed to
 melt the present ice. Then, the surplus energy is heating the
 sea water, setting the surface temperature to
\be
{\di T_s\, =\, T_f\, -\frac{\rho_i\, L_i\, h_i}{\rho_w\, c_{p_s}\, h_{mix}} },
\ee
with $h_i < 0$.

\section*{Ice formation, freezing and melting}
If the surface temperature of open ocean water is below the freezing point,
sea ice is formed. The heat flux available for freezing is given as

\be
Q_f\, =\, {\di \frac{\rho_w\, c_{p_w}\, h_{ml}}{\Delta t} \, (T_s\, -T_f)}
\ee

where $\rho_w$ is the density of sea water,
$c_{p_w}$ is the specific heat of sea water and
$h_{ml}$ denotes the mixed layer depth. The thickness of the new 
formed ice sheet is calculated by setting $Q\, =\, Q_f\, +\tilde{Q}$
in (\ref{hieq}). Since the model differentiates only between 
no ice and full ice, a
minimum ice thickness $h_{i,min}$ (default = 0.1m) needs to be present 
before a grid point is treated as ice covered (compactness $A$  = 1). 
If $h_i$ is less than $h_{i,min}$ the heat flux $Q_a+\tilde{Q}$ is used 
to build (or melt) ice. If $h_i > h_{i,min}$ a ice surface temperature 
($T_i$) is computed (see below) and, if $T_i < T_f$, ice growths according 
to $Q_c + \tilde{Q}$. Ice is diminished if the ice surface temperature 
would be above freezing point. 


\section*{Sea ice temperature}
If a grid point is covered by sea ice (i.e.~$h_i > h_{i,min}$) a sea ice surface temperature $T_i$ is calculated from the energy balance
at the ice surface. To avoid numerical problems (due to large changes of $T_i$ within one time step), the energy balance equation is solved for an upper layer of the ice/snow column which has a heat capacity, $c_p^{\star}$, similar to that of $h_{i,min}$ pure ice ($c_p^{\star}=h_{i,min} c_{p_i} \rho_i$):

\be
{\di c_p^{\star}
\frac{\partial T_i}{\partial t} -Q_b = 0 \Rightarrow 
\frac{\partial T_i}{\partial t} = 
\frac{Q_b}{c_p^{\star}}}
\label{ti_eq}
\ee

where $Q_b\, =\, Q_a\, +Q_c$ with $Q_a$ as defined in (\ref{qa_eq}) and $Q_c$ from (\ref{qc_eq}). Eq.~(\ref{ti_eq}) is solved using an implicit formulation for the conductive heat flux $Q_c$.

\section*{Snow cover}
If a grid point is covered by sea ice, snow fall is accumulated on top of the ice. Snow cover effects the albedo and the heat conductivity (according to eq.~(\ref{eq_kap})). Snow is converted to sea ice if there is sufficient snow to suppress the ice/snow interface below the sea level. The conversion conserves mass. The new ice ($h_i^{new}$) and snow ($h_s^{new}$) thicknesses are given by:

\begin{eqnarray}
h_i^{new} & = &\frac{\rho_s h_s + \rho_i h_i}{\rho_w}  \\
h_s^{new} & = & \frac{\rho_w - \rho_i}{\rho_s} h_i^{new}  
\end{eqnarray}

Where $\rho_w$ and $\rho_s$ are the densities of sea water and snow, respectively.

If the surface temperature
is above freezing point, first the snow is melted, then the ice. Snow melts
according to
\be
{\di \frac{dh_s}{dt}\, =\, \frac{Q_a}{\rho_s\,L_{sn}},}
\ee
where $\rho_s\, \rm (kg/m^{3})$ is the density of snow and
$L_{sn}\,\rm (J/kg)$ is the latent heat of fusion of snow.
If the atmospheric
heat flux is so large that it melts all the snow, then the remaining energy
melts ice via (\ref{hi_eq}). 

\btbh
\begin{tabular}{lccl}
\hline
Parameter & Symbol & Value & Reference \\
\hline
density of sea ice		& $\rho_i$	& $\rm 920\, kg\, m^{-3}$		& \citp{Kiehl et al.}{1996, p. 139} \\
density of snow			& $\rho_s$	& $\rm 330\, kg\, m^{-3}$		& \citp{Kiehl et al.}{1996, p. 139} \\
density of sea water$^a$	& $\rho_w$ 	& $\rm 1030\, kg\, m^{-3}$		&  \\
latent heat of fusion (ice)	& $L_i$ 	& $\rm 3.28\times 10^{5}\, J\, kg^{-1}$		& \citp{Kiehl et al.}{1996, p. 139}\\
latent heat of fusion (snow)	& $L_{sn}$ 	& $\rm 3.32\times 10^{5}\, J\, kg^{-1}$		& \citp{Kiehl et al.}{1996, p. 139} \\
heat conductivity in ice	& $\kappa_i$ 	& $\rm 2.03\, W\, m^{-1}\, K^{-1}$	& \citp{Kiehl et al.}{1996, p. 139} \\
heat conductivity in snow	& $\kappa_s$ 	& $\rm 0.31\, W\, m^{-1}\, K^{-1}$	& \citp{Kiehl et al.}{1996, p. 139} \\
specific heat of sea ice	& $c_{p_i}$ 	& $\rm 2070\, J\, kg^{-1}\, K^{-1}$	& \citp{Kiehl et al.}{1996, p. 139} \\
specific heat of snow		& $c_{p_s}$ 	& $\rm 2090\, J\, kg^{-1}\, K^{-1}$	& \citp{Kiehl et al.}{1996, p. 139} \\
specific heat of sea water 	& $c_{p_w}$ 	& $\rm 4180\, J\, kg^{-1}\, K^{-1}$	&  \\
freezing point of seawater $^a$	& $T_f$ 	& $\rm 271.25\, K$			&  \\
ocean water salinity		& $S_w$ 	& 34.7 psu				&  \\
\hline
\end{tabular}
\caption[]{Thermodynamic parameter values.\\
$^a$ at S=34.7
}
\label{iceparatab}
\etb
\nocite{apel1987}
\nocite{kiehl1996}
\nocite{king1997}








